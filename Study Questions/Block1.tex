\section{Study questions block 1}
\subsection{\textbf{1. What does calibration mean in the context of electrical measurement systems, and why is it important?}}
To calibrate a system is to readjust the output to fall in line with a known input value. This is done by changing internal values or models to fall in line across a designated range. 

Without calibration the system will give skewed outputs and be unreliable. 

\subsection{\textbf{2. Describe the basic steps in a typical calibration process.}}
Input a known variable into the system, adjust the values of the system to align with the calibrator. Rinse and repeat until a satesfactory accuracy is achieved.

\subsection{\textbf{3. What is the difference between accuracy and precision in a measurement system?}}
Accuracy is how close to a true value a measurement is.

Precision is how big your error of the same measurement is.
\subsection{\textbf{4. Explain the concepts of traceability and uncertainty in measurement and how they affect calibration.}}
Traceability is the practice of noting the environmental and time aspects of any measurements to account for any such factors. 

Every system has some level of uncertainty and you are not able to calibrate the system more accuratly than the systems uncertainty. 

\subsection{\textbf{5. What are the common reasons why a measurement system needs calibration?}}

Any system which are not self-calibrated will drift and most will be dependant on environmental factors. These things will need to be adjusted for and said process is called calibration.


\subsection{\textbf{6. What is a calibration standard, and why is it important to use standardized references?}}
A set of guidelines and procedures to calibrate a system which are often tailor made for the system in question. 

By having an official way to calibrate a system it allows comparisons be made across systems without having to dive into what system of calibration said system used.

\subsection{\textbf{7. Describe how a multimeter is calibrated and which parameters are checked during the process.}}
A multimeter is connected to a reference Voltage/resistance/current, measure it using the multimeter and adjust the internal setting of the multimeter to align with the refernece.

\subsection{\textbf{8. How is measurement uncertainty affected if a measurement system is not regularly calibrated?}}
The uncertainty of a system will continue to deviate with time.

\subsection{\textbf{9. What is a “zero point calibration,” and in what situations is it used?}}
Whenever a system is calibrated along the zero value of the system. A system with no input ought to ouput 0 and all other measurements can be judged from it. (assuming the system is correctly linearly calibrated)

\subsection{\textbf{10. Explain the difference between static and dynamic calibration of measurement systems.}}
A static calibration is done with a non-changing reference and gains you a single data point. (At x input -> Y output)

A dynamic calibration is done with a changing reference and gains you a graph worth of data. (At X input -> y(X) output)

\subsection{\textbf{11. What is the difference between absolute calibration and relative calibration?}}
A relative calibration is calibrated against a certain calibration reference and can differ from other relative calibrated systems depending on the calibration reference and the environmental factors when it was done.

An absolute calibration correlates the values given of a system with a physical constant or other directly observable or mathematically derived values.


\subsection{\textbf{12. Provide examples of how environmental factors, such as temperature and humidity, can affect the calibration of electrical measurement systems.}}

If the temperature deviates the measuring components will react differently to electrical values. 

Other physical values are also dependant on temperature and the like. The density of a material will change along with temperature while the boiling point of liquids are dependant on the preassure.


\subsection{\textbf{13. What role do calibration certificates play, and what should be checked on a calibration certificate?}}


\subsection{\textbf{14. Explain how an uncertainty budget is established for a measurement system and its role in calibration.}}


\subsection{\textbf{15. What is the “calibration interval,” and how are suitable calibration intervals determined for a measurement system?}}
The calibration interval is how often a system needs to be recalibrated to ensure a level of certainty. This interval is dependant on the fastest deviating component. 
