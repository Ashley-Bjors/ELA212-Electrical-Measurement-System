\section[short]{Bitwise operations}

\subsection[short]{What is the value of the 'result' variable at each occasion it is assigned a value?}

\begin{center}
    \begin{tabular}{ |c|c|c|c| } 
        \hline
        Operator &  op1 & op2 & result\\
        \hline
        AND1 & 0000 0001 & 0000 0011 & 0000 0001\\
        AND2 & 0000 0011 & 0000 0111 & 0000 0011\\
        OR & 0000 0001 & 0000 0011 & 0000 0011\\
        XOR & 0000 0001 & 0000 0011 & 0000 0010\\
        NOT & 0000 0001 & NaN & 1111 1110\\
        R SHIFT & 0000 0010 & NaN & 0000 0001\\
        R SHIFT x4 & 1000 0000 & NaN & 0000 1000\\
        L SHIFT & 0000 0001 & NaN & 0000 0010\\
        L SHIFT x7 & 0000 0001 & NaN & 1000 0000\\
        Logic AND & 0000 0001 & 0000 0011 & TRUE\\
        Logic OR & 0000 0001 & 0000 0011 & TRUE\\
        \hline
    \end{tabular}
\end{center}
\subsection[short]{What happens at the end of the program?}
A while(TRUE) loop is inserted to keep the program from returning from main since there is no OS to handle a return. 